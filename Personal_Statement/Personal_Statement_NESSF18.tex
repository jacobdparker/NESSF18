\documentclass[10pt,letterpaper]{article}
\usepackage[latin1]{inputenc}
\usepackage{amsmath}
\usepackage{amsfonts}
\usepackage{amssymb}
\usepackage{graphicx}
\begin{document}
	\title{Jacob Parker Personal Statement}
	\date{}
	\maketitle
	
	Throughout my career as a student I have had countless opportunities. Thanks to the tremendous faculty at MSU and collaborating institutions I have been able to design and manufacture experimental components, develop code for solar physics data analysis and modeling, collaborate with Marshal Space Flight Center on the ESIS sounding rocket, work as a member of the IRIS team as a science planner, preform optical testing and assembly of EUV telescopes, and attended many conferences throughout the country to present my work.  This variety is what makes me excited about my future career in Solar Physics
	
	As an Undergraduate I took the first couple years of the Mechanical Engineering curriculum.  After two years the differences in the Physics and Engineering Departments were shining through and I had to pick a lane, Physics.  Before I left the ME department behind I was able to learn enough about mechanical design and machining to build simple parts.  This knowledge allowed me to be useful as a member of the MSU Sounding rocket team as a Sophomore.  Since then my role on the team has expanded significantly.  As a Physics Graduate Student I manage the ``mechanical'' portion of the sounding rocket team which includes maintaining the schedule, managing several undergraduates and one professional engineer, as well as contributing to opto-mechanical design and testing.  As we start to run out of ``mechanical'' tasks I have had the opportunity to learn how to use our 4-D interferometer and theodolites for optical testing, focus, and alignment.  Working on the sounding rocket team gives me plenty of opportunities to get away from the computer and ``turn a wrench". I hope to continue working on solar telescopes in the future in order to contribute new and exciting data to the community.
	
	Last spring I had the opportunity to travel to Lockheed Martin in Palo Alto to participate in IRIS science planner training.  Since then I have planned a month in total and have learned a lot about IRIS and its operations.  My experience interacting with the professional scientists on the IRIS team has been invaluable in preparing me to be a productive member of the IRIS science team.  With one IRIS related publication completed I look forward to digging further into this state of the art dataset.
	
	Going forward with my career I hope to continue to use and develop spectral instruments to diagnose solar magnetic reconnection.  Being able to develop completely theoretical MHD models, design and build a new mirror mount, and do large numerical studies of data, all for one project is what drew me to solar physics, and it is what will keep my passion alive in the future.
	
	
\end{document}